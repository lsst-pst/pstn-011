\section{Introduction}

intro goes here. 

\section{Science Raft Towers}
Unlike most astronomical imagers, the Large Survey Synoptic Telescope (LSST) mosaic camera will be constructed from 21 identical modules, known as “rafts,” Science Rafts, or Raft Tower Modules (RTMs). This choice is motivated by the large area of the focal plane (3217 cm$^2$ unvignetted field) and issues of integration, test, and maintainability. The alternative option, a monolithic structure supporting the entire complement of sensors and electronics, leads to difficulty accessing the central portion of the array during construction and repair. Each raft can be an autonomously operated as fully functional (and testable) camera. In view of the long production time for the charge-coupled device (CCD) sensors, series production and testing of the rafts streamlines and reduces risk for the project. By maintaining a set of hot spares, the mean time to repair the camera in the event of sensor or electronics failures can be minimized. A diagram of the focal plane layout is shown in Figure \ref{fig:Fig1}.
 
\begin{figure}[htbp]
\begin{center}
\includegraphics[width=6in]{./figures/PSTN011_Figure1}
\caption{Focal plane organization: science sensors (blue), guide sensors (yellow), wavefront sensors (green), Science Raft boundaries (red).}
\label{fig:Fig1}
\end{center}
\end{figure}
 
\section{Science Raft Requirements}
The Science Raft Tower Module must provide: 
\begin{itemize}
\item{} Precise mechanical location and support for the CCD array 
\item{} Electronic support functions to control and read out the CCDs
\item{} Thermal management of the CCDs and electronics.
\end{itemize}

Mechanical requirements are extremely stringent and include locating the CCD imaging surfaces to within a narrow flatness envelope, with gaps of 0.150 mm between adjacent CCDs on a single raft and 0.250 mm between CCDs on adjacent rafts. As a result of the fine segmentation of the CCDs, the RTM has to implement all electronic functions of a 144-channel CCD controller in a volume contained within the projected footprint of the Raft. Although low-power application specific integrated circuits (ASICs) are used, the total power dissipation from an RTM is significant, and that power must be removed from the cryostat through adequate thermal paths. Finally, the imaging surfaces of the CCDs must be protected from contamination by condensable materials in the electronics enclosures as well as conductive materials contacting the exposed electronic connections.

\section{Science Raft Tower Module Design Description}
The Science Raft, shown in Figure \ref{fig:Fig2}, is the modular building block of the camera focal plane and consists of three major sub-assemblies:
\begin{enumerate}
\item{}The Raft Sensor Assembly (RSA) – A 3 x 3 mosaic of science CCDs in precision packages, mounted on a silicon carbide baseplate. The RSA also includes temperature sensors, resistive heaters, and flexible electrical and thermal connections.
\item{}The Raft Electronics Crate (REC) – Houses three circuit boards and contains copper planes providing the thermal path for cooling the RSA and the CCDs.
\item{}Three Raft Electronics Boards (REBs) – Contains custom video processing and clock/bias buffering electronics plus video digitizing, clock sequencing, bias generation, temperature sensing, and interfacing to the control and data acquisition systems. 
\end{enumerate}

All components of the RTM are contained within the camera cryostat vacuum space. The RSA and REC are maintained at an operating temperature of around $-$100\degsym C. This temperature is chosen as a compromise between sensor dark current and near infrared (NIR) quantum efficiency (QE). The REBs operate at a warmer temperature, to reduce the load on the cryostat cooling system and to allow the use of standard commercial electronics. The power dissipated on the REBs is extracted via a coldplate that is held at \
$-$40\degsym  C. To prevent volatile organic contaminants (VOCs) from condensing on the CCD imaging surfaces several precautions are taken: The circuit boards and components are tested for low outgassing of volatiles, and CCDs are never allowed to be the coldest surfaces in the cryostat. 
 
\begin{figure}[htbp]
\begin{center}
\includegraphics[width=6in]{./figures/PSTN011_Figure2}
\caption{(top) Exploded and (bottom) perspective views of LSST science raft .}
\label{fig:Fig2}
\end{center}
\end{figure}

\section{Raft Tower Electronics}
The function of the Science Raft Electronics is to fully support the nine CCDs that reside on a raft. It must provide all biasing and clocking signals, amplification and low noise analog signal processing, digitization, data collection, and transmission of image and metadata to the camera’s Data Acquisition (DAQ) system. Finally, it must provide flexible timing and control in a manner that is precise, predictable, and programmable under control of the Camera Control System. Some key performance parameters are summarized in Table 10-1.

%Table 10-1: Key Science Raft electronics specifications
%Specification 	Value 
%Readout time 	2 seconds  
%
%Read noise 	9 electrons RMS 
%
%Blooming full well capacity 	Not to exceed 175,000 electrons 
%
%Dynamic range 	16 bits 
%
%Digitization	18 bits
%
%Crosstalk 	2x10-3  
%
%Gain stability over one hour
%0.1\% 
%
%Linearity	3\% 
%
%Focal plane temperature measurement absolute value 	$\pm$1.5C 
%
%Focal plane temperature measurement stability 	$\pm$0.15C 
%
%Focal plane temperature resolution (least significant bit) 
%less than or equal to 0.05 C 
%
%Contribution to image quality, full width half maximum (FWHM)	0.3 arcseconds
%Vertical range for 95\% of the pixels on a raft	$\pm$10 $\mu$m relative to height datum

Each of the sensors in the Science Raft comprises 4K x 4K pix, or 16 Mpix total. To simultaneously meet the read time and read noise specifications, the sensors have been divided into 16 segments of 1 Mpix, each with its own output amplifier. This allows readout to proceed at an average rate of ~550 kpix/s, a rate which has been shown to be compatible with the noise requirement. This segmentation results in 144 parallel streams of image data for each raft. The raft dimensions of 127 mm x 127 mm results in extremely tight packaging requirements for the 144 parallel channels of analog signal processing, analog to digital conversion, and data collection. 
The REBs also have the responsibility for monitoring and reporting a variety of metadata such as supply voltages and currents, precision temperature measurements, and operating raft heaters to maintain focal plane temperature. In addition, a significant ability to diagnose problems in-situ is part of the design. 
Each Science Raft contains a 3 x 3 array of science sensors and forms an integral unit along with its accompanying Science Raft electronic package. The focal plane consists of 21 such Science Raft and four Corner Rafts (described in Chapter 11). 

\subsection{Electronics architecture}
The functionality of the Science Raft electronics is contained on three REBs which hold the analog functions and digital electronics that provide clocks and biases to three CCDs and handle the 48 channels of video output from those CCDs. Each REB also provides power regulation, monitoring of voltages and currents and conversion of the 48 streams of digitized video data into a single high speed link that is connected to the DAQ system as shown in Figure 10-3. 
 
\begin{figure}[htbp]
\begin{center}
\includegraphics[width=6in]{./figures/PSTN011_Figure3}
\caption{Raft Electronics Board block diagram.}
\label{fig:Fig3}
\end{center}
\end{figure}

A significant feature of the electronics is that each raft, and indeed each REB, can function as an autonomous readout and control system for nine (or three) CCDs. The raft and three REBs can function as a fully autonomous 144 Mpix camera, when connected to an LSST-style DAQ system.
To mitigate potential crosstalk and pickup noise issues, all 21 Science Rafts in the focal plane operate synchronously. The requisite synchronicity is maintained by distributing a master clock for the entire focal plane. A real time microcontroller, implemented in a Xilinx field-programmable gate array (FPGA) on each REB, fans out timing and gating signals to the custom ASICs and commercial analog-to-digital converters (ADCs) at an accuracy of order a few nanoseconds and with a jitter less than 100 ps. 
Focusing on the electronics, a block diagram of the overall mechanical assembly is shown in Figure \ref{fig:Fig4}.

\begin{figure}[htbp]
\begin{center}
\includegraphics[width=6in]{./figures/PSTN011_Figure4}
\caption{Block diagram of overall Science Raft electronics .}
\label{fig:Fig4}
\end{center}
\end{figure}
  
\subsection{Electronics physical location}

The physical location of the main electronic systems is a significant issue which is dependent on a number of important factors. One factor is how much, if any, of the electronics should reside within the cryostat, and how much should be placed outside. 
The pin count on the sensors themselves is an important consideration. Each of the science sensors has of order 150 bond pads, each of which requires connection to the readout electronics. There are thus in excess of 30,000 bond pads in the focal plane requiring connections and consequently one motivating factor in determining the electronics location is the desire to avoid this large number of vacuum feedthroughs in the cryostat. Secondly, we know that a large contributor to crosstalk is in the cable connection between the sensors and electronics. Each sensor cable has multiple video channels with conductor to conductor capacitance. That this capacitance is a source of crosstalk implies that video cables should be as short as possible and for signal amplification to be located as close to the focal plane as possible. Further, transmission of analog signals from the first amplification point should be differential if they are to span any reasonably long distances. Thus, at a minimum, we require over 6,000 differential video connections exiting the focal plane. 
Other connections required for clock driving, biasing, and power, quickly escalate the total to well over 10,000 connections. In addition, given the optical design of the telescope, the electronics package must reside in the shadow of the focal plane. Given those factors, the LSST camera design has located the full readout chain inside the cryostat – minimizing sensor cable length and vacuum penetrations at some cost for additional in-situ electronics cooling. Given the large channel density, analog amplification and low noise signal processing requires ASIC technology. 

\subsection{Custom Analog Electronics}
The analog front end of the electronics system is responsible for providing clock signals and bias voltages to the CCDs and then taking the voltage signals from the CCD output amplifiers through an amplification and correlated dual sampling stage to prepare the CCD video signals for digitization. The video chain functions are accomplished by a custom integrated circuit – the Analog Signal Processing Integrated Circuit (ASPIC), which provides preamplification and correlated double sampling for eight channels of CCD output.
In the original design, the clock drivers and positive bias generators were implemented as a second custom ASIC, the Clock and Bias ASIC for CCD (CABAC). However, design problems in the second version of the CABAC necessitated a significant redesign effort which, while largely successful, did not provide schedule margin. During the chip redesign effort, a backup solution using commercial parts was created and prototyped. As the commercial solution was shown to fully satisfy all LSST requirements, was adopted in the final design. 
Both the ASPIC and the commercial clock and bias drivers are mounted close to the CCDs on the REB custom printed circuit serving three CCDs – six ASPICs and three bias/clock drivers per REB. The other significant block of the electronics chain, also near the CCDs,  are the 16 current sources which bias the on-CCD source follower transistor. These are implemented using a degenerated junction field-effect transistors (JFETs) whose current can be programmed from 1 – 5 mA. The ASPIC has as its goal to amplify and process the signal coming from the focal plane CCDs in order to optimize the signal-to-noise ratio before digitization. 
Each ASPIC contains eight channels and is implemented in the AMS 0.35 $\mu$m 5 V CMOS process. The chip operates down to -100C and provides processed differential signals to the commercial 18-bit ADC.
The architecture chosen for the LSST camera signal processing is a Dual Slope Integrator (DSI). DSI processing combines correlated double sampling with CCD signal integration. Correlated double sampling, by eliminating reset noise, removes kT/C and 1/f noise while integration removes thermal noise. A block diagram of one ASPIC IV (ASPIC version 4) channel is shown in Figure 10-5. 

 \begin{figure}[htbp]
\begin{center}
\includegraphics[width=6in]{./figures/PSTN011_Figure5}
\caption{ASPIC custom integrated circuit channel block diagram.}
\label{fig:Fig5}
\end{center}
\end{figure}

The correlated double sampling is performed by integrating first the CCD reference signal just after reset and then the charge signal with opposite polarity during the same integration time as shown in Figure \ref{fig:Fig6}. The difference of these two integrations provides the CCD signal with the kT/C and 1/f noise contributions suppressed.

 \begin{figure}[htbp]
\begin{center}
\includegraphics[width=6in]{./figures/PSTN011_Figure6}
\caption{Dual slope integration sequence.}
\label{fig:Fig6}
\end{center}
\end{figure}

The CCD will be coupled to the ASPIC through an external capacitance. A large clamp switch performs a direct current (DC) level restore regularly; this large switch will provide a very low resistance, around a few hundred Ohms.
The ASPIC has been designed to meet the following requirements as part of the overall electronics requirements:
\begin{itemize}
\item{} Operates as low as $-$100\degsym C, with required running temperature  50 – 27\degsym C. Includes a temperature probe.
\item{} Noise: At 500 ns integration time for an input of 5.75 $\mu$V/e- and a gain of 1.9 : less than  2.3 e- (requested),  with a goal of 1.8 e-.
\item{} Operation at 0.1 – 1 Mpix/s, nominally 550 kpix/s.
\item{} 0.05\% maximum crosstalk between channels at 550 kpix/s, with a goal of 0.01\%.
\item{} Gain: 1.7 – 6, for nominal gain of 1.9 compatible with a 180k e- full well capacity for 5.75 $\mu$V/e-.
\item{} 0.5\% linearity requested , with a goal of 0.3\%.
\item{} Output load: 25 pF in parallel with 1 k$\Omega$.
\item{} Differential output: $\pm$2.5 V from central reference voltage.
\item{} Operating Voltage: 5 V with central reference of 2.5 V.
\item{} Power Dissipation: less than or equal to 38 mW/channel.
\end{itemize}
The ASPIC implements a power down feature which allows a significant reduction in dissipated power during non-readout periods (below 2 mW used per channel for the ASPIC in idle mode). Use of this feature is not part of the REB baseline.

\subsection{Raft Electronics Board (REB)}
The REB that services a set of three of the CCDs on a raft is a complex 16 layer board with blind and buried vias about 104 mm x 419 mm x 2.5 mm with surface mounted active and passive components on both surfaces. The layout of the production REB is shown in Figure \ref{fig:Fig7} with component groupings illustrated CCDs, RSA temperature monitors, and  RSA trim heaters connect via 8 miniature connectors on the left edge of the board, power and high-speed digital I/O enter on two connectors on the right. On the left half of the board one can identify the analog and ADC blocks replicated in three identical “stripes” that each serve one of the three CCDs served by the REB.
 
\begin{figure}[htbp]
\begin{center}
\includegraphics[width=6in]{./figures/PSTN011_Figure7}
\caption{Layout of the production REB. Thermal and mechanical components not shown.}
\label{fig:Fig7}
\end{center}
\end{figure}
 
\subsection{Digitization and Control Electronics}
The output of the ASPIC is sent as a differential analog signal to a commercial differential buffer and 18 bit ADC for digitization. Rated for 1 MS/s, the ADC is run at the LSST pixel rate of 550 kpix/s. This pixel data is fed as 48 individual serial streams directly into the FPGA where the firmware formats the data into packets to be sent out via the high speed electro-optical link to the back-end DAQ hardware in the LSST Observatory Control Room. At the same time, sequencer logic in the firmware provides direct control of the timing of all the circuitry involved in the video chain – parallel and serial CCD clocks, clamps, resets, ramp up and ramp down for the correlated double sampling in the ASPIC and analog-to-digital conversion – all based upon a common 100 MHz clock distributed from the DAQ hardware to each REB. In addition, there are numerous “housekeeping” tasks such as providing programmable voltages required by the CCD for biasing, monitoring temperature sensors distributed on the REB, raft and sensor packages, and preparing meta-data packets for transmission to the DAQ. Each of the “housekeeping” devices is controlled by an independent serial bus from the FPGA so that it is possible to simultaneously read a precision temperature from a sensor-mounted resistance temperature detector (RTD),set a new clock rail voltage and also monitor the power supply currents.
In addition to supporting the analog video chain and housekeeping circuits, the REB also provides:
\begin{itemize}
\item{} A hardware serial number
\item{} DACs and power operational amplifiers (op amps) to control heaters on the RSA
\item{} Precision 5 V and 2.5 V references for the video chain
\item{} DACs and power op amps to provide rail voltages for the parallel, serial and reset clocks
\item{} Heavy internal ground planes plus externally mounted copper bars to carry heat from the REB to the coldplate
\item{} Auxiliary ADC channels to measure bias voltages, drain currents and power regulator currents and voltages
\item{} Protection circuitry to allow the back substrate bias to be clamped to ground until a logic signal is sent from the FPGA
\item{} Diode reverse voltage protection on all sensitive CCD nodes.
\end{itemize}

In May 2016, the basic functionality of the REB was demonstrated at the three-CCD level, with ASPIC III and CABAC0 of the ASICs. The commercial clock driver circuitry was successfully demonstrated using a modified corner raft board in June of 2015. The production version (REB5) final design and manufacturing readiness review was completed October 2016.

\subsection{Electronics powering}
The rafts will operate in vacuum with conduction cooling to the coldplate and cryoplate, whose heat is removed by a mixed-refrigerant refrigeration system. Therefore, a key design goal was to minimize heat generation on the CCDs and REBs; achieving this constrained power budget was a strong motivation for adopting analog correlated double sampling rather than a digital implementation. Rafts require six low voltage supplies (+5 V, $\pm$15 V, +7 V, and +40 V) and the CCD substrate bias at -70 V. Average power dissipation is about 14 W per REB (42 W per raft, 882 W total for science rafts in the Camera).

\section{Science Raft Tower Module Mechanical Design and Analysis}

The raft tower serves several purposes. It is the mechanical structure that houses and protects the REBs which interface with the CCDs; it serves as the heat sink path to the cryofluid plate; it helps to prevent VOCs liberated from the electronics from contaminating the CCD image surface; and, it provides an initial platform on which the RSA sub-assembly can be supported before inclusion in the camera cryostat (see Figure \ref{fig:Fig8}). The tower concept allows the design of the focal plane to be modular, and it is a replaceable unit within the cryostat. There are 21 science Raft Tower Modules (RTMs) in the focal plane.
 
\begin{figure}[htbp]
\begin{center}
\includegraphics[width=6in]{./figures/PSTN011_Figure8}
\caption{Detail of Raft Sensor Assembly (RSA).}
\label{fig:Fig8}
\end{center}
\end{figure}

The tower has a smaller “footprint” than the RSA. While the RSA is 126.5 mm on each side, with the CCDs forming a 127.0 mm square, the tower envelope forms a rectangle 116.0 mm x 117.0 mm. The larger dimension includes the extremities of the spring hold-downs. Note: The grid bay opening is 120.0 mm square. The RTMs total height is 476 mm from the focal plane to its extremities, before installation in the cryostat (final installation of the tower increases this dimension by 4 mm, as the RSA is mated with the kinematic supports on the grid and the REC walls are pulled up off the temporary supports and mated with the cryoplate). Each of the 21 RTMs has a total mass of 9.8 kg.
The materials for the RTM must satisfy several requirements. Of prime importance is excellent mechanical stability over a temperature range of $-$100 to  +20\degsym C, high thermal conductivity, and low thermal expansivity. In addition, low-temperature strength and ductility, good machinability, low density to reduce static and dynamic mass, and availability in small production volumes are also desirable.

\subsection{Interaction between Tower and Raft/CCD Sub-assemblies}
The tower is connected electrically and thermally to the RSA, as shown in Figure 10-9, but it must remain uncoupled from it mechanically as much as possible so that mechanical distortions of the image plane are avoided. Electrically, the mechanical isolation is accomplished by using polyimide film pigtails with copper traces between the CCDs and REBs. Pigtails are permanently affixed or manually connected to the CCD, with a nano-connector at the other end. 
From the time of RTM construction and until installation in the cryostat is complete, the tower is the sole support for the RSA. During this period, features on the REC walls engage the underside of the baseplate so that there exists a stable, reproducible position of the RSA bearing against the tower in all axes. This is especially important during the RTM installation into or removal from the cryostat in order to avoid collisions with CCDs from previously installed RTMs, as they come in very close proximity.

\subsection{Thermal Connections}
The total electrical power dissipated on each RTM is approximately 60 W. This comes from four sources: radiated heat from lens L3 onto the focal plane; amplifiers and clocks on each CCD; makeup heaters that regulate the RSA temperature; and, heat generation from electronic functions on the REB.  Nearly all the heat on the REB (operating between 
$-$20 – +8\degsym C), is removed via conduction into copper bars attached to the REB, and then through semi-flexible straps to the coldplate maintained at -40 C. A small amount of heat leaks into the RSA via the flexible circuits and the heater wires. The heat on the RSA (operating at or near $-$100\degsym C) which is from the CCD amplifiers and clocks, radiation from lens L3, and the heat leak from the electronics via the flexible circuit shown in Figure 10-8, is removed by flexible conductors, and then transferred by conduction through the REC walls, then to the cryoplate maintained at $-$130\degsym C, and finally into the cryofluid.
The REC walls are thick copper plates to allow good heat transfer from the flexible thermal straps on the RSA to the bolted interface at the cryoplate. Bolted connections will have Belleville and lock washers to accommodate thermal contraction and cyclic thermal ratcheting effects.
 
\begin{figure}[htbp]
\begin{center}
\includegraphics[width=6in]{./figures/PSTN011_Figure9}
\caption{Thermo-mechanical mounting details for RSA and RTM.}
\label{fig:Fig9}
\end{center}
\end{figure}

The components on the REBs generate approximately 180 W of power (all three boards) that is removed through the cooling bars into the coldplate. The board material is an FR4/copper sandwich to aid in lateral heat conduction. The cold bars are attached to the coldplate with semi-flexible straps. Mechanical compliance is obtained by using sliding clamped joints between each board at its edges. 
Finite element analysis (FEA) modeling of the temperature distribution across the REB during operation (shown in Figure \ref{fig:Fig10})was generated using a prototype REB design, and results matched measured REB2 values within 5\degsym C. 
 
 \begin{figure}[htbp]
\begin{center}
\includegraphics[width=6in]{./figures/PSTN011_Figure10}
\caption{FEA model of REB thermal design, showing equilibrium temperature distribution for coldplate held at $-$40\degsym C.}
\label{fig:Fig10}
\end{center}
\end{figure}

\subsection{Science Raft Hold-Down Design}
The RSA is designed to be supported in the cryostat on a three-ball kinematic mount. The balls and their cup mounts are an integral part of the cryostat support grid, whose lattice-like structure forms “bays” into which each RTM is installed. V-grooves on the three mounts of the baseplate engage these balls. A total spring load equivalent to 5g is imposed on the RSA to assure unchanging contact pressure in all camera attitudes and slewing motions, with this load being applied directly over the center of each ball to eliminate bending forces within the raft. The mechanical sub-assemblies designed to apply this load are called the “raft hold-downs.” There are two hold-downs per RTM, positioned on two opposite sides of the tower, with springs sustaining the hold-down force (see Figure \ref{fig:Fig11}).
While there must be the inherent electrical and thermal connections from the RSA to the tower, it is advantageous to mechanically uncouple the RSA from the remainder of the RTM in order to eliminate as many sources of image plane distortion as possible. For this reason the forces of the hold-down springs are not allowed to remain reacted by the tower (as they are during RTM construction), but are instead transferred to the support grid after the RTM is installed on the cryogenically-cooled cryoplate. Once decoupled, the REC position is independent of the RSA position. Furthermore, neither static deflection of the grid due to weight nor thermal motion during transient cool-down can induce any effect on the position or shape of the RSA on the kinematic mount after this decoupling is accomplished.
To transfer the spring load from tower to grid, each hold-down employs a captive screw. There is very limited access to the area inside each grid bay and total inaccessibility from the image plane side of the RTM, so the screw is actuated through cut-outs provided in the cryoplate. Rotating each screw the specified number of turns compresses the spring the appropriate amount and completes the load transfer from REC to grid bay wall. 

\begin{figure}[htbp]
\begin{center}
\includegraphics[width=6in]{./figures/PSTN011_Figure11}
\caption{Hold-downs on RTM.}
\label{fig:Fig11}
\end{center}
\end{figure}

The maximum stress in either hold-down is on the back-side of the arm. The highest localized von Mises stress in this location is around 81 MPa (yield strength for the material is ~240 MPa). The average shear stress on the pin attaching the hold-down to the RSA is 28 MPa.
10.4.5	Thermal and Mechanical Analysis
ANSYS FEA software has been used along with many hand calculations and spreadsheets to determine the thermal and mechanical response of the RTM to the loads imposed on it. Analysis simulations have been run for thermal steady state and transient behavior and static structural and modal/vibration. The model analyzed is a robust compilation, including all materials and interfaces between components within the complex assembly. Where applicable, and without reducing accuracy, some simplifications and de-featuring have been implemented for enhancing convergence and reducing run time. The simplifications involve eliminating features (such as threads) and parts insignificant to the thermal, stress, and deflection responses, and combining some mated parts into new single parts where possible. However, the simplified model itself is not all that simple, primarily due to the fact that the mechanical supports and thermal paths of the CCDs are not mirror symmetric about the transverse axes. Therefore, the entire RTM must be modeled to obtain accurate results. 
The thermal path from the focal plane to the heatsinks is through the CCD sensor supports. Extensive testing characterized the thermal contact impedance, and later verified through FEA analysis. Analytical assumptions and results matched well before and after testing, and have been refined for the high-fidelity thermal model in use. An illustration is provided in Figure 10-12. 

\subsection{The RTM FEA Model}
The cryofluid passage surfaces in the plate are set to -130 C, consistent with the fluid properties. The coldplate temperature is set to -40 C, also consistent with the properties of that fluid.
%Imperfect conductivity across mating part faces is accounted for in the ANSYS software by manually entering conductance values based on the materials mated. The applied values have been taken from various published sources, as well as from the results of the contact resistance testing performed, and represent average-to-conservative values as shown in Table 10-2.
%Table 10-2: Contact conductance values across various part faces, between the cryoplate and RTM interfaces
%Value [W/m$^2$~C]
%Applied Joints
%
%Infinite
%soldered joints
%10,000	Cu to Cu, Si to Si, etc. with good contact
%
%2000
%Cu to SS and all bolted connections
%50002
%Cu to CeSiC
%
%25003
%CeSic to CeSiC
%10003
%Cu to G-10

%1 Apiezon-N or Indium applied as interposer; 2 Based on data for Cu/Si joint, no interposer; 3 estimate

The heat inputs to the focal plane from lens L3 plus CCD chip functions (applied to surface), and from CCD amplifiers (applied to actual locations of amplifiers within the CCD substrate), are indicated in Figure \ref{fig:Fig12}; and heat inputs to one REB are shown from the FPGA side in Figure \ref{fig:Fig13}. The other side of the REB does not have an FPGA and all other components are mirrored. Note that the commercial components which replaced the CABAC dissipate less heat. 

\begin{figure}[htbp]
\begin{center}
\includegraphics[width=6in]{./figures/PSTN011_Figure12}
\caption{Thermal model, focal plane inputs.}
\label{fig:Fig12}
\end{center}
\end{figure} 

\begin{figure}[htbp]
\begin{center}
\includegraphics[width=6in]{./figures/PSTN011_Figure13}
\caption{Thermal model, REB inputs.}
\label{fig:Fig13}
\end{center}
\end{figure} 

\subsection{Steady State Thermal Analysis}
From an initial temperature of 22\degsym C, the model cools to an equilibrium condition governed by the cryoplate, the heat inputs, the materials chosen, and the conductances assigned. The resulting overall temperature profiles of the RSA sub-components are shown in Figure \ref{fig:Fig14}. The imaging surface shows an expected temperature gradient, cooler toward the middle, reflecting the cooling paths connected toward the center underside of the part. The maximum and minimum temperatures from simulating the focal plane surface are $-$98.6\degsym C and $-$100.7\degsym C, respectively. This is consistent with the goal temperature of $-$100\degsym C, and with the allowed temperature differences across the focal plane. The temperature profiles of the tower and its components (REBs, thermal straps, REC walls) are shown in Figure \ref{fig:Fig15}. Again, one sees the cooler REB regions near the coldplate attachment points with a maximum $\Delta$T along the REB of about 30\degsym C.

\subsection{Static Structural Analysis}
This analysis was run to determine the deflection of the focal plane due to gravity and the hold-down forces alone. It gives a baseline for comparison with the deflection that includes thermal effects. The gravity direction is applied perpendicular to the focal plane surface and directed away from it, simulating the camera pointing toward earth.
Shown in Figure \ref{fig:Fig16}, the flatness of the sensor imaging surface is 119 nm, with the crown of the bow in the center area, parallel to the baseplate ribs underneath, as expected. Each CCD is bowed individually about 640 nm, which masks the raft bowing to a large degree. The larger amount of bowing in the CCDs is primarily due to the single unsupported corner of each CCD. All of these deflections meet specification.
Looking at raft motion from the underside in Figure \ref{fig:Fig17} shows that the kinematic mounts as modeled are properly allowing raft contraction to occur along the V-grooves. In addition, the stress plot shows that the Von Mises stress at these grooves is very low. Both of these conditions are important to verify that deflections in the raft are occurring for reasons that are appropriate, not by improper over-constraint. This model presented does not include the frictional effects between the mounting balls and the V-grooves.
Figure 10-18 shows the deflection of the Raft surface with the thermal contraction and the mechanical response superimposed. The resulting flatness is well within tolerance (.013 mm requirement). The total height change of ~.005 mm can be accommodated in the design.

\begin{figure}[htbp]
\begin{center}
\includegraphics[width=6in]{./figures/PSTN011_Figure14}
\caption{Modeled RSA temperatures.}
\label{fig:Fig14}
\end{center}
\end{figure}

\begin{figure}[htbp]
\begin{center}
\includegraphics[width=6in]{./figures/PSTN011_Figure15}
\caption{Science Raft Tower steady-state temperatures).}
\label{fig:Fig15}
\end{center}
\end{figure} 
 
\begin{figure}[htbp]
\begin{center}
\includegraphics[width=6in]{./figures/PSTN011_Figure16}
\caption{Gravity deflection from FEA analysis.}
\label{fig:Fig16}
\end{center}
\end{figure}

\begin{figure}[htbp]
\begin{center}
\includegraphics[width=6in]{./figures/PSTN011_Figure17}
\caption{Kinematic mounts.}
\label{fig:Fig17}
\end{center}
\end{figure}
 
 \begin{figure}[htbp]
\begin{center}
\includegraphics[width=6in]{./figures/PSTN011_Figure18}
\caption{Raft surface deflection, mechanical plus thermal.}
\label{fig:Fig18}
\end{center}
\end{figure}
 
\section{Science Raft Tower Module (RTM) Assembly}

The LSST camera project plan calls for the RTMs to be assembled and tested at BNL, and then shipped to SLAC for integration into the LSST instrument. This section describes the plans for the assembly of RTMs. 

\subsection{Raft Tower Assembly and Test Facility at BNL}
A ~1750 ft2 ISO 7 cleanroom has been built in the BNL physics building (see Figure \ref{fig:Fig19}).

\begin{figure}[htbp]
\begin{center}
\includegraphics[width=6in]{./figures/PSTN011_Figure19}
\caption{Raft assembly cleanroom design.}
\label{fig:Fig19}
\end{center}
\end{figure} 

Construction funds were provided by Scientific Laboratories Infrastructure (SLI) program within the DOE Office of Science for the construction and revitalization of general purpose infrastructure as part of a series of major lab and office space renovations that will improve facilities in the Physics and Chemistry departments at BNL. 
Acceptance testing of the CCDs, construction of the 25 RSAs, metrology tests, integration with raft electronics, and final raft testing has been performed within this facility. In order to meet schedule objectives for science raft construction over a 2-1/2 year funding period, the facility will have two dedicated single CCD-test stations, two metrology stations, two raft assembly stations and a station for conducting qualification tests of a fully integrated RTM. The room is separated by an ESD-protective curtain to partition the space and offer an environment that is better controlled due to limited activities in that area. This space contains almost all the work requiring sensor surface exposure to air. A laminar flow hood is to be used during testing and assembly steps which require the most amount of exposure, i.e. RSA assembly. 

\subsection{Science Raft Assembly Sequence}
Providing a detailed description of the assembly process is outside the scope of this chapter. What follows are the highlights of the major test and assembly steps, as outlined in Figure \ref{fig:Fig20}.
The assembly procedure begins by populating the raft with nine CCDs acceptable for LSST: a statistically sampled group of sensors which were previously checked for metrology and electro-optical performance. During assembly and testing of the RSA and RTM, protective tooling is applied to encase the CCDs and other sub-components from damage. Specialized measuring equipment and dedicated tooling is then used to check the mosaic flatness of the nine CCDs on the RSA before continuing the construction of the full RTM. The protective tooling also provides mounting features throughout assembly and testing.
The two thick copper tower walls are the paths for heat removal from the RTM. The two walls orthogonal to them complete the mechanical structure. To assure final coplanarity of the mounting surfaces of the heat path copper walls to the cryoplate, all four walls are pre-assembled and held in position relative to each other via precision dowel pins in the corners, then the copper interface surfaces for the cryoplate are pre-machined as an assembled unit. The walls are then disassembled but are kept together as a set. The dowel pins, asymmetrically placed to allow reassembly of the four walls in only one orientation, guarantee proper mating to the cryoplate.
The Raft Tower Module (RTM) assembly uses specialized tooling to hold and manipulate the RTM sub-assemblies during construction, such as the RSA, REC walls and REBs. The polyimide flexi-cables and their integral nano-connectors are accessible in the fixture. The nano-connectors are mated to the REBs in a prescribed order to best facilitate access to each connector. The REC and other components are assembled around the REBs. The REB attachment to the REC utilizes an attachment system to allow for thermal expansion, as the REBs are at a different temperatures than the REC walls. The distal end of each hold-down is attached to each of the three raft mounting points, followed by the completion of each hold-down assembly to the REC walls. Each hold-down mechanism is adjusted to provide a defined pre-load between the REC and the RSA assembly using a temporary support structure. After the RTM performance is measured and qualified, it is placed into a custom designed environmentally controlled crate that is used for shipment and long-term storage.

\begin{figure}[htbp]
\begin{center}
\includegraphics[width=6in]{./figures/PSTN011_Figure20}
\caption{Abbreviated Science Raft assembly sequence.}
\label{fig:Fig20}
\end{center}
\end{figure}

\section{Corner Rafts}

Four special purpose rafts, called corner rafts, are mounted at the corners of the LSST science Camera,
and contain wavefront sensors and guide sensors. A single design can accommodate two guide sensors
and one split-plane wavefront sensor integrated into the four corner locations in the camera. Although
contained within the LSST Camera, the corner rafts have a broader role in the operation of the
telescope. The location of the corner rafts and their sensors are shown in Figure \ref{fig:Fig21}.

\begin{figure}[htbp]
\begin{center}
\includegraphics[width=6in]{./figures/PSTN011_Figure21}
\caption{Corner raft locations, showing wavefront sensors (green) and guiders (yellow).}
\label{fig:Fig21}
\end{center}
\end{figure}

\subsection{Corner Raft Requirements}

The location of the corner raft is dictated by the requirement that the wavefront sensor cover 4 field
angles in a square geometry with the highest possible atmospheric de-correlation in order to
reconstruct the phase at each mirror. The guide sensors have a similar requirement. This has driven the
corner raft to be located at the edge of the field and therefore at the edge of the sensor array. Thus the
corner raft sub-system has a triangular cross section as opposed to the square cross section of the
science raft. Similar to the science rafts, many of the design challenges relate to packaging the required
functionality in a tightly constrained volume.
Another driver is related to the leveraging of the electronics system and infrastructure (such as thermal
management and mechanical stability) from the science raft development effort. The challenge for the
corner raft is to re-package the electronics chain used for the science sensor CCDs such that it
accommodates the corner raft geometry while maintaining the logical functionality of the design.
Interfaces to the cryo plate and cold plate and the contamination control features of the corner raft also
have to be similar to the science raft.
The precision required in the placement of the sensors has been shown to be similar to that required for
the science sensors and therefore most of the design principles and mechanisms are similar to those
used in the science raft.

% corner raft figures start as Figure21
\subsection{Corner Raft Design}

The corner raft sub-system is responsible for acquiring curvature wavefront data for the active optics
system. The corner raft must acquire intra and extra focal images at 4 locations at the edges of the focal
plane that will be used by the Telescope Control System to estimate the wavefront and, through
reconstruction, provide inputs to the active optics controller. Images are collected in parallel with the
science images and follow the corresponding exposure time and cadence (nominally 15 second exposure
and 2 second readout time). The current specification requires flexibility to adjust the intra to extra focal
separation up to 4.8mm from the nominal 4mm separation. This will be set at the time of final assembly
based on final analysis and tests.

The guide sensor specifications have been set to ensure that the servo loop jitter for point spread
functions on the sky better than 0.6 arc-sec FWHM will be less than 0.02 arc-sec FWHM. Based on the
servo-loop implementation, this requirement can also be expressed as a single guide sensor rms
centroid accuracy of 0.04arc-sec = 0.2 pixels with worst case flux of 800e- (u-band)

%Table 11-1: Wavefront sensor specifications
%Key requirement Value
%Pixel Size 10 $\mu$m square pixels
%Readout Noise 10 e- rms
%Intra/Extra focal sensor separation 4 mm $\pm$-0.1 mm
%Integration time 15 s
%Readout Time 2 s
%Sensor Flatness/z-axis tolerance $\pm$15 $\mu$m
%The corner raft sub-system is also responsible for acquiring small images centered on selected bright
%stars at a nominal 9Hz rate. These images are used to estimate the pointing variation during the
%exposure and provide input to the telescope control tracking system. The guider system interface
%specifies the performances of the guide sensors as described in Table 11-2.
%Table 11-2: Guide sensor specifications
%Key requirement Value
%Pixel Size 10 $\mu$m square pixels
%Readout Noise 9 e- rms
%Integration time 50 ms
%Frame Rate 9 Hz
%Window size 50 x 50 pixels
%Sensor Flatness/z-axis tolerance $\pm$30 $\mu$m
%Number of Guide Sensors 8

\subsection{Corner Raft Architecture within the LSST Camera}

The corner raft subsystem will provide images from 8 guide sensors and 4 wavefront split sensors. The
images are collected at the focal plane by those sensors. The sensors are readout by the analog
electronics located in the corner raft towers. The electronics in each corner raft tower reads the data
from the sensors, processes and digitizes the signals, and transmits the digitized data to the data
acquisition (DAQ) systems over a high speed serial interface (PGP link). The DAQ provides access to the
wavefront images to either the Telescope Control System (TCS) for close loop operation or to the data
management system for archival storage. The DAQ provides access to the guide star image data and the
guide system is responsible for computing the star centroid location and estimating the changes as they
are measured at 9Hz. The telescope decides whether the information is used to close the loop or not,
depending on the current mode of operation. The Observatory Control System (OCS) initiates
commands resulting in images being acquired by the corner raft sub-system. The OCS coordinates with
the Camera Control System (CCS) for camera specific actions including the corner raft operating steps
and the TCS for active control. The conceptual block diagram of the corner raft system in shown in Figure \ref{fig:Fig22}.

\begin{figure}[htbp]
\begin{center}
\includegraphics[width=6in]{./figures/PSTN011_Figure22}
\caption{Corner raft system component block diagram.}
\label{fig:Fig22}
\end{center}
\end{figure}

\section{Corner Raft Tower Mechanical Design and Analysis}

\subsection{Corner Raft Sensor Assembly}
Each sensor package will be mounted to the corner raft structure by means of 3 threaded studs + 2 pins,
Belleville washers and locking nuts. Removable, thermally-conductive spacer/shims will be used to set
the height of the sensor surfaces above the corner raft. Identical corner rafts will be mounted in four
corner locations on the GRID structure (which also supports the science rafts) by means of an adjustable
3-point ball-and-vee kinematic mount design. The concept for this mount system is shown in Figure \ref{fig:Fig23}, 
and the integration of these into a corner raft is shown in Figure \ref{fig:Fig24}.

\begin{figure}[htbp]
\begin{center}
\includegraphics[width=6in]{./figures/PSTN011_Figure23}
\caption{Side view of sensor-to-corner-raft grid mounting scheme.}
\label{fig:Fig23}
\end{center}
\end{figure}

\begin{figure}[htbp]
\begin{center}
\includegraphics[width=6in]{./figures/PSTN011_Figure24}
\caption{Guide and wavefront sensors on corner raft baseplate.}
\label{fig:Fig24}
\end{center}
\end{figure}

Two guide sensors are located in each of the 4 corner rafts (total of 8 guide sensors). The baseline
design for guiders uses 4K x 4K LSST science CCD sensors with 10 micron pixels. The CCD sensors are
deep-depletion, back-illuminated devices with a highly segmented architecture. The readout electronics
are mounted on a custom printed circuit board and housed inside a metal enclosure for EMI shielding
and heat transfer to the cold plate. A pair of flat flex cables (identical to those used connectivity in the
science CCD array) mates to circuitry on the back-side of the CCD sensor package, and terminates in a
pair of 37-pin Nano-connectors for connection to the readout electronics board.

LSST requires four wavefront sensors located in the corner rafts at the periphery of the focal plane. A
depiction of the wavefront sensor concept is shown in Figure \ref{fig:Fig25}. Two 2K x 4K segmented CCDs
packages will be mounted on a “step plate” and offset relative to the science focal plane to produce
intra and extra focal images. An offset of roughly $\pm$2mm from the science focal plane is anticipated.
Wavefront sensor CCDs will share the layout and processing of the science sensors, and have similar
performance as science CCDs regarding quantum efficiency, point spread function, read noise, and
readout speed. Detailed mechanical specifications such as fill factor, flatness, parallelism of the intra
and extra focal sections, and temperature sensing will be validated by laboratory measurements.

\begin{figure}[htbp]
\begin{center}
\includegraphics[width=6in]{./figures/PSTN011_Figure25}
\caption{Wavefront sensor assembly.}
\label{fig:Fig25}
\end{center}
\end{figure}

The wavefront sensor mechanical alignment is tied to the science sensor array best fit plane as it is a
critical component of wavefront correction. The details of the overall mechanical alignment
requirements applicable to the corner raft are shown in Figure \ref{fig:Fig26}.

\begin{figure}[htbp]
\begin{center}
\includegraphics[width=6in]{./figures/PSTN011_Figure26}
\caption{Corner raft sensor mechanical alignment and tolerance requirements.}
\label{fig:Fig26}
\end{center}
\end{figure}

The back side of the corner raft features three radially-arrayed vees ground into the Raft material. The
vees uniquely define the position of the corner raft with respect to silicon-nitride ceramic balls that are
fixed in cups mounted on the grid. This forms a kinematic connection that isolates the corner (and
science) rafts from grid distortion due to external dynamic and transient loads, assembly tolerances, and
to the expected thermal motions due to differential contraction during cooling. The corner rafts are
held in position by springs that pre-load the corner rafts against their kinematic coupling to the grid,
producing a uniform, invariant loading. The deflection of grid and corner raft due to the spring loading is
compensated for during initial integration, and remains unchanged in operation, independent of camera
orientation and temperature.
The grid, science rafts, and corner rafts will be manufactured from a silicon-carbide ceramic matrix
composite. This material is used for lightweight and stable space-borne structures, including focal
planes, optical benches, and mirrors. The high modulus, high thermal conductivity, near-zero expansion
coefficient and improved fracture toughness of the ECM CeSic material makes it ideal for the grid,
science raft, and corner raft structures. A closed-loop thermal control system will be used to adjust the
temperature of the corner rafts, thereby maintaining a suitable and stable operating temperature for
the sensors.

\subsection{Corner Raft Tower Concept}

Each corner raft tower contains one wavefront sensor and two guide sensors and dedicated electronics.
The mechanical and thermal design of the corner rafts is as similar as possible to the science rafts.
The electronics for operating the wavefront and guide sensors are packaged within the volume behind
the detectors, similar to the science raft configuration, in a corner raft electronics crate (CREC) “Tower.”
Electrical connections between sensors and front-end electronics boards are made by flat flex circuit
cables. The electronics boards are thermally connected to the cold plate at about -40C by thermally
conductive straps. The grid supports as little mass as possible—only the sensor rafts and their
supports—to reduce gravity induced deflections. The tower is supported by a neighboring cryogenically-cooled
Cryo Plate at about -130C temperature. Each corner raft is thermally connected, but structurally
decoupled, from the tower by use of thin copper thermal straps to remove the electrical and radiant
heat load from each corner raft and conduct it back to the cryoplate. Details of the design for corner
raft/towers for the LSST camera are shown in Figure \ref{fig:Fig27}.

\begin{figure}[htbp]
\begin{center}
\includegraphics[width=6in]{./figures/PSTN011_Figure27}
\caption{CAD model of the corner raft tower.}
\label{fig:Fig27}
\end{center}
\end{figure}

\subsection{Corner Raft Mechanical Interfaces}

\subsubsection{Corner Raft Cryo Plate Interface}

The corner raft tower is a mechanical enclosure that contains the readout electronics boards for the
wavefront and guide sensors and has a smaller “footprint” than the corner raft. The tower is fastened
to the cryo plate by a series of screws to provide mechanical support as well as the thermal path from
corner raft CCDs to the cryo plate. The CRT also acts as a barrier to contaminants outgassed from
electronics that could degrade the image quality of the CCDs (see Figure \ref{fig:Fig28}).
The corner raft WREB (wavefront readout electronics board) and GREB (Guider readout electronics
board) are passed through a triangular-shaped opening in the cryo plate during installation of the corner
raft tower module (CRTM) in the cryostat. Access to the captive screws for the two corner raft hold-down
mechanisms is through cut-outs in the cryo plate adjoining the opening for the WREB and GREB.

\begin{figure}[htbp]
\begin{center}
\includegraphics[width=6in]{./figures/PSTN011_Figure28}
\caption{Corner raft to cryo plate interface.}
\label{fig:Fig28}
\end{center}
\end{figure}

\subsubsection{Corner Raft Cold Plate Interface}

The corner raft tower is a mechanical structure that houses the readout electronics boards. Thermal
connections are made between cold bars on the WREB and GREB and cold plate "fingers" during
installation of CRTMs into the cryostat (see Figure \ref{fig:Fig29}). Copper blocks at the ends of copper braided
wire cables (connected to the cold bars on the REBs) are retracted within the profile of the REBs during
insertion of the CRTMs into the cryostat, then extended and fastened to the cold plate fingers.

\begin{figure}[htbp]
\begin{center}
\includegraphics[width=6in]{./figures/PSTN011_Figure29}
\caption{Corner raft to cold plate interface.}
\label{fig:Fig29}
\end{center}
\end{figure}

\subsubsection{Corner Raft Cryostat Interface} 

The designs for the mechanical and thermal interfaces between the corner rafts and the cryostat are as
similar as the possible to the interface designs for the science rafts. The mechanical interface between
the corner raft and the grid provides precise mechanical location and support for the corner raft sensors
with respect to the focal plane. The mechanical and thermal interfaces between the cryostat and corner
raft towers support the electronics to control and read out the CCDs and provide the thermal paths for
removal of the heat from the corner raft CCDs and electronics without adversely affecting the
performance of the science raft subsystems. This is shown in Figure \ref{fig:Fig30}.

\begin{figure}[htbp]
\begin{center}
\includegraphics[width=6in]{./figures/PSTN011_Figure30}
\caption{Corner raft to cryostat interface.}
\label{fig:Fig30}
\end{center}
\end{figure}

\subsubsection{Corner Raft Thermal and Structural Analysis}

Thermal steady-state finite element analysis results of a simplified corner raft tower model, with
average REBs power (23.5W), 2.7W heat load on the corner raft sensor assembly, and temperatures at
the cryo plate interface fixed at $-$130\degsym C and at the cold plate interface fixed at $-$40\degsym C, are shown in Figure
\ref{fig:Fig31}. Makeup heat of $\sim$0.5W from ohmic heaters mounted on the backside of the corner raft are
required to raise the temperature of the sensors to $-$100\degsym C. A closed-loop thermal circuit will be used to
adjust the make-up heat on each corner raft to maintain a stable operating temperature for the sensors
with varying thermal loads.

\begin{figure}[htbp]
\begin{center}
\includegraphics[width=6in]{./figures/PSTN011_Figure31}
\caption{Corner raft tower thermal analysis results.}
\label{fig:Fig31}
\end{center}
\end{figure}

Preliminary structural finite element analysis of a simplified corner raft and sensors model, with the
corner raft kinematically supported by the vees on fixed balls, and including thermal loads, gravity and
applied spring forces, results in a ``z-axis" deformation of the corner raft and sensor surfaces of ~1 $\mu$m,
meeting requirements. The structural analysis results are illustrated in Figure \ref{fig:Fig32}. 

\begin{figure}[htbp]
\begin{center}
\includegraphics[width=6in]{./figures/PSTN011_Figure32}
\caption{Corner raft structural analysis results.}
\label{fig:Fig32}
\end{center}
\end{figure}

\section{Corner Raft Tower Electronics}

Data acquisition and control for the wavefront and guide sensors are managed using the same
infrastructure as for the Science detectors. The corner raft tower contains four ASPIC ASICs for reading
the two guide sensor data and two ASPIC ASICs for the split wavefront sensor. The ASPIC implements
the CCD readout and was developed specifically for LSST. The electronics used to read the CCD sensors
are schematically nearly identical to the science electronics and are simply re-packaged to
accommodate the smaller footprint and reduced numbers of channels.

\subsection{Guide Sensing Electronics Architecture}

The corner raft tower electronics provides all CCD biasing and clock signals, CCD signal amplification and
low noise analog signal processing, digitization, data collection, and transmission of image and metadata
to the DAQ. The corner raft tower electronics also has the responsibility for monitoring and reporting a
variety of metadata such as supply voltages and currents, precision temperature measurements, and
operating raft heaters to maintain the focal plane temperature. In addition, a significant ability to
diagnose problems in situ will be part of the design. The video data from the CCDs will be transmitted
via two high speed serial links (one for both guide sensors, one for the wavefront sensor) to the DAQ
system. Analogous to the dual functionality of the corner raft (guiding and wavefront sensing) the
electronic is partitioned into 2 PCBs: one for the operation of the two guide sensors at high frame rates
with a total of 32 analog channels (GREB for Guider Readout Electronics Board) and one for the
operation of the wavefront sensor with 16 channels (WREB for Wavefront Readout Electronics Board).
The two different CCD types (for guide sensor and wavefront sensor) will be very similar in terms of
segmentation and readout and therefore will be interchangeable with respect to the electronics.
Therefore it is foreseen to utilize the same basic electronics block for both PCB types. Still, some
optimization for the specific functionality will be implemented, especially to support the fast shift and
region of interest readout for the guide sensors. The CCDs will be connected to the readout boards via
polyimide flex leads. Length and layout of this connection is critical for noise and crosstalk performance.

\subsubsection{Guide Sensor Electronics}
The guide sensor electronics is based on the science sensor electronics and is responsible for providing
clock signals and bias voltages to the CCDs as well as processing and digitizing the CCD output signals.
The design uses the ASPIC integrated circuit to amplify and process the signal from the CCDs. The
required CCD bias voltages and clock waveforms are generated by the DCBS block. The CCD back-side
bias is generated by an external power supply but a dedicated switch to enable this voltage is included
on the board. The circuitry is identical to the science sensor design, but the implementation is repackaged
to fit the diagonally shaped corner raft tower. The guide sensor electronics services 2 CCDs
and therefore is organized with two video stripes (each stripe using two ASPICs and 16 ADCs).
Additionally, during operation the sequencing electronics is operated at 9Hz to meet the guide servo
loop requirements.

The two guide CCDs are serviced by an electronics board as shown in Figure \ref{fig:Fig33}.

\begin{figure}[htbp]
\begin{center}
\includegraphics[width=6in]{./figures/PSTN011_Figure33}
\caption{Guider readout electronics block diagram.}
\label{fig:Fig33}
\end{center}
\end{figure}

Even though the 2 guide sensors located in each corner raft are controlled using a single board, their
sequencing will be concurrent but different to accommodate different guide windows locations. Each
guide sensor board also provides temperature sensing near various components on the board. Support
for temperature monitoring and heating of the corner raft is also available in a similar fashion to the
science REB capabilities. Power supplies, bias voltage, as well as clock signal filtering are also available.
The FPGA firmware is reconfigurable via the DAQ interface. Register content in the FPGA is accessible via
the GDS interface and this interface is used to handle the instruction set updates and meta-data
transfers (a few tens of words) defining CCD and ASPIC clock structure.
Guide CCD operations are controlled by a synchronous readout processor implemented within the FPGA
of the guide sensing electronics. The microcode for this processor is downloaded through the DAQ so
that the processor is completely unaware of the purpose of any particular code. This architecture is
particularly useful in the case of the guide sensor for two reasons. First, the same electronics design
used in the science sensor electronics can be re-used in the corner raft with only changes to the
downloaded firmware needed. Second, the windowed operation can be simplified or upgraded by
loading new code to the electronics as needed.
The guide sensing CCD sequences are more complicated than science raft sequences, but it is expected
that the code would be the same except for guide window coordinates. A possible guider sequence
implementation compatible with the overall guider timing is given below:

\begin{itemize}
\item{}Set up the sequence microcode without guide window coordinates
\item{} Download window coordinates to registers
\item{} Execute single frame read with large window
\item{} Download smaller guide window
\item{} Execute frame reads until told to stop
\end{itemize}

\subsection{Guide Sensing Electronics Interfaces}

\subsection{Power Supply Interface}

As shown in the table below, the power distribution system (PDS) provides a number of different
potentials to the guide sensing back-end electronics in the corner raft – similar voltages as used in the
science rafts, but at lower maximum current. Each supply is floating and is separately adjusted and
monitored. The adjustment, monitoring and on/off sequencing of the supplies is controlled by CCS.
The PDS has separate power supply groups for each of the 21 science rafts and for each of the four
corner rafts. In general, the power used in the corner raft, as in the science rafts, is regulated at the
point of load to further reduce interference – for instance the ASPIC uses a separate low dropout linear
regulators to take A$\_$VDD from about 6V to the 5.0V and 3.3V required by the chip. Bias and clock rail
voltages are regulated with OPAMPs and emitter follower output transistors.

%Table 11-3: Power supply voltages
%Voltage(V)
%(nom.)
%Current(A)
%(peak)
%Name Description
%6.0 1.5 V_ANA Analog power
%5.0 1.0 V_DIG Digital power
%5.0 1.0 V_HTR Power for raft heaters
%+14 (e2V)
%+8 (ITL)
%0.8 V_CLK_H Auxiliary power for clock rail
%generation
%-2 (e2V)
%-8 (ITL)
%0.8 V_CLK_L Auxiliary power for clock rail
%generation
%32 .13 V_OD CCD bias generation power
%12 0.7 V_dPHI CCD parallel clock amplitude
%-70 .002 V_bias Adjustable 0 to -70V – CCD bias
%11.5.3.2

\subsection{Corner Raft DAQ Interface} 

The guide sensing electronics interface to the DAQ System through via their Source Communication
Interface (SCI) firmware block. The corner raft is by default in reset mode, where the pixels are
continuously read and their charge reset. After the shutter opens, the raft initiates readout of 50x50
pixel windows to be read at 9Hz. Based on the location in the CCD segment, the readout time can range
from 1.25 msec to 35 msec and the fixed 50 msec integration time is accounting for this range. It has
been shown that, as long as the start of integration on all sensors is synchronized within 1 msec, the
servo loop rejection ratio and bandwidth is appropriate for guiding. All windows are read out such that
their integration start times are all synchronized within 1msec. As data are read out by the raft, they are
forwarded to the telescope for centroid finding computations and servo loop control.

\subsection{Wavefront Sensing Electronics Architecture}

The wavefront sensors electronics is physically independent from the guide sensor electronics and has
their own independent board; however, they are co-located within the mechanical crates. The
wavefront sensor electronics architecture is nearly identical to that for readout of each of the two guide
sensors on the GREB. The wavefront sensor electronics operates independent of the guider electronics.
For electronics readout purposes, the split wavefront sensor of two 2kx4k pixels is seen as a unique
4kx4k sensor. This allows the electronics to operate fully synchronously with the science electronics as
they share the same integration time. This guarantees that the potential cross-talk and pickup noise
issues are limited and similar to the science sensor operation so that the identical cross-talk correction
algorithm and processing chain can be used.

\subsection{Wavefront Sensing Readout Electronics}

As mentioned previously, the wavefront sensing electronics is based on the science sensor electronics
and is responsible for providing clock signals and bias voltages to the CCDs as well as output signal
processing and digitization. The circuitry is practically identical to the guide sensor electronics but the
implementation is re-packaged to fit the diagonally shaped corner raft tower and simplified to a single
strip due to the need of controlling a single equivalent 4kx4k sensor. In contrast to the guide sensor
electronics, this board will not include trim-heater circuitry.
The split CCD sensor, constructed of two identical CCDs, in a corner raft tower is serviced by a wavefront
readout board (WREB) as shown in Figure \ref{fig:Fig34}.

\begin{figure}[htbp]
\begin{center}
\includegraphics[width=6in]{./figures/PSTN011_Figure34}
\caption{Wavefront sensing electronics block diagram.}
\label{fig:Fig34}
\end{center}
\end{figure}

\subsection{Wavefront Sensing Electronics Interfaces}

\subsubsection{Power Supply Interface}
The power distribution system provides a number of different potentials to the wavefront sensing
electronics in the corner raft. Those voltages are identical to those provided for the guider.

\subsubsection{Timing Interface}
The wavefront sensor timing is generated by the FPGA on the WREB and the timing programs are loaded
via the DAQ system. The wavefront sensing FPGA operates identically to the science sensor FPGA as the
readout is done at the same time with the same timing.

\subsubsection{DAQ Interface}
The interface between wavefront sensing electronics and the DAQ System is identical to the interface
between the Science REB and the DAQ System.

\section{Corner Raft Assembly and Test}

Given that LSST is a survey telescope, the mechanical and electrical systems must be reliable, with a high
up-time. All mechanical and electrical sub-assemblies will be functionally and performance tested prior
to installing in the corner raft towers, minimizing the risk of early failures and the need for disassembly
during subsequent camera integration and commissioning. Sensors and electronic sub-assemblies are
modular and self-contained, allowing them to be installed, removed, and serviced with minimal
disturbance of neighboring components.
After assembly and system testing, each corner raft tower will be integrated into the LSST Camera. The
same procedure and installation tooling on a gantry motion control system envisioned for integration of
science raft towers will be used for integration of corner raft towers into the Camera cryostat.

\subsection{Corner Raft Assembly Sequence}

Corner raft-specific assembly fixtures and procedures are adaptations from science raft assembly.
Sensors will be assembled into corner rafts in a face-down orientation to allow easy access to the
installation mechanism and to minimize the potential for surface contamination and accidents.
Installation rods are screwed onto the ends of two of the threaded stud legs on the sensor package.
Using a handling fixture, the sensor package with flex cables attached is pulled into contact with the
corner raft. The taper on the rods is designed to fully engage the mounting holes in the corner raft
before the sensor package overlaps any neighboring sensors. Figure \ref{fig:Fig35} shows the 
corner raft assembly jig. 

\begin{figure}[htbp]
\begin{center}
\includegraphics[width=6in]{./figures/PSTN011_Figure35}
\caption{Wave front and guide sensor insertion tooling.}
\label{fig:Fig35}
\end{center}
\end{figure}

After sensors are installed and leveled in the corner raft, a protective lid will be fastened to the corner
raft sensor assembly (CRSA) to protect the sensors from particulates and contamination. The thermal
straps, readout electronics, tower sides, and spring-loaded yoke hold-down sub-assemblies are
progressively connected to the CRSA (see Figure \ref{fig:Fig36}). A conductance barrier (to separate the ultra-clean
focal plane from the support electronics) is formed during assembly by the installation of bar-shaped
parts between thermal straps and sensor flex cables across the front end of the tower.

\begin{figure}[htbp]
\begin{center}
\includegraphics[width=6in]{./figures/PSTN011_Figure36}
\caption{Corner raft tower assembly sequence.}
\label{fig:Fig36}
\end{center}
\end{figure}

\subsection{Corner Raft Installation into the Camera}

Corner raft towers are modular and designed to use the same tooling and insertion, mounting,
connecting, and testing processes developed for installation of the science raft towers into the cryostat.
As with the integration of science raft towers, corner raft towers are inserted into the corner bays of the
grid and cryo plate from below, with the cryostat pointed downward in its integration stand. An
insertion arm on the XYZ-gantry is extended down through the cryostat and fastened to the back-end of
the corner raft tower. The corner raft tower is pulled into the cryostat through the grid; making small
adjustments with the XYZ-gantry to maintain alignment during insertion (see Figure \ref{fig:Fig37}).

\begin{figure}[htbp]
\begin{center}
\includegraphics[width=6in]{./figures/PSTN011_Figure37}
\caption{CRTM installation into the cryostat.}
\label{fig:Fig37}
\end{center}
\end{figure}

As the corner raft reaches its mounting location, V-blocks on the corner raft contact the balls on the
grid, lifting the corner raft off the temporary supports on the tower. The tower is pulled an additional
$\sim$ 4mm into the cryostat until its back end contacts the cryo plate. The same pre-loaded spring hold-down
mechanism design for the science rafts is used to allow this relative motion between the corner
raft and tower. The tower is the fastened to the cryo plate with screws, and the insertion arm is
disconnected from the back of the tower and retracted from the cryostat.
Mechanical installation is completed by transferring the load of the hold-down springs from the sides of
the Tower to support points on the back of the grid. This load transfer is accomplished by engaging the
captive screws on the grid into the hold-down mechanisms to lock the corner raft into place on the grid.
Following mechanical integration, the corner raft tower is electrically integrated by connecting the
vacuum feedthrough pigtail connectors to the back of the electronics boards. Testing using equipment
connected to the pigtails on the outside of the cryostat will be performed to verify that all internal
connectors are mated as well as ensure that all corner raft tower systems are functional.

